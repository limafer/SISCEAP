
    \documentclass[12pt,a4paper]{article}
    \usepackage[utf8]{inputenc}
    \usepackage{amsmath}
    \usepackage{amsfonts}
    \usepackage{amssymb}
    \usepackage{graphicx}
    %\usepackage{tikz}
    %\usepackage{tikz-3dplot}
    \usepackage{enumerate}
    %\usetikzlibrary{intersections}
    \usepackage[left=2.0cm,top=1.5cm,right=2.0cm]{geometry}
    \newcommand{\edo}[3]{$#1\,{\rm{d}}x#2\,{\rm{d}}y=#3$}
    \thispagestyle{empty}
    
    \everymath{\displaystyle}
    

\begin{document}
\begin{center}
\large LISTA DE EXERCÍCIOS DE GEOMETRIA ANALITICA\\[1mm]
{\large\bf Departamento de Matemática}\\[1mm] \textit{Prof. Lindeval: lindeval.ufrr@gmail.com}\\\end{center}

\noindent\rule{17.0cm}{0.7mm}\\[0.5cm]
\noindent
1) Dada a reta $r: x = 2t - 1$, $y = 1 + t$, $z = -3t$, e o plano $x + 2y - z = 4$, determine se a reta é paralela ao plano.\\[2mm]
\noindent
2) Explique o que são coordenadas homogêneas e como elas são usadas na geometria computacional.\\[2mm]
\noindent
3) Determine o vetor resultante de $\vec{AB}$ se $A(1, 2)$ e $B(-3, 0)$.\\[2mm]
\noindent
4) Seja $\vec{u} = (1, 2, 3)$ e $\vec{v} = (-1, 0, 2)$. Calcule o produto vetorial $\vec{u} \times \vec{v}$.\\[2mm]
\noindent
5) Determine um vetor ortogonal a $\vec{v} = (3, -2)$.\\[2mm]
\noindent
6) Seja $\vec{u} = (1, -1)$ e $\vec{v} = (3, 2)$. Calcule a projeção ortogonal de $\vec{v}$ sobre $\vec{u}$.\\[2mm]
\noindent
7) Seja $\vec{u} = (3, -2)$ e $\vec{v} = (-1, 4)$. Calcule a soma $\vec{u} + \vec{v}$.\\[2mm]
\noindent
8) Dê a definição de vetores colineares e coplanares.\\[2mm]
\noindent
9) Escreva a equação geral de um plano.\\[2mm]
\noindent
10) Explique como determinar a equação do plano dado um ponto no plano e um vetor normal.\\[2mm]
\noindent
11) Determine a equação do plano que passa pelo ponto $P(2,-1,3)$ e é perpendicular ao vetor $\vec{n} = (1,2,-1)$.\\[2mm]
\noindent
12) Determine a equação do plano que contém os pontos $A(1,2,3)$, $B(-1,3,5)$ e $C(2,-1,4)$.
\\[2mm]
\noindent
13) Seja a reta $r: x = -2 + 3t$, $y = 1 - t$, $z = 4 + 2t$, e o plano $x - 2y + z = 5$. Determine o ponto de interseção.\\[2mm]
\noindent
14) Dada a reta $r: x = 2t - 1$, $y = 3t + 2$, $z = -t$, e o plano $2x - y + 3z = 4$, determine o ponto de interseção.\\[2mm]
\noindent
15) Determine o ângulo entre a reta $r: x = 1 + t$, $y = -2 - t$, $z = 3t$ e o plano $2x - y + z = 1$.\\[2mm]
\noindent
16) Escreva a equação vetorial e paramétrica de uma reta no plano.
\\[2mm]
\noindent
17) Seja a reta $r$ dada por $x = 2t + 1$, $y = -3t - 2$, $z = t$. Determine um ponto e um vetor diretor para essa reta.
\\[2mm]
\vfill\hfill\bf{\textit{Boas Atividades!}}
\end{document}