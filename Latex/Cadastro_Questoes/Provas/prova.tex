
    \documentclass[12pt,a4paper]{article}
    \usepackage[utf8]{inputenc}
    \usepackage{amsmath}
    \usepackage{amsfonts}
    \usepackage{amssymb}
    \usepackage{graphicx}
    %\usepackage{tikz}
    %\usepackage{tikz-3dplot}
    \usepackage{enumerate}
    %\usetikzlibrary{intersections}
    \usepackage[left=2.0cm,top=1.5cm,right=2.0cm]{geometry}
    \newcommand{\edo}[3]{$#1\,{\rm{d}}x#2\,{\rm{d}}y=#3$}
    \thispagestyle{empty}
    
    \everymath{\displaystyle}
    

\begin{document}
\begin{center}
\large AVALIAÇÃO DE GEOMETRIA ANALITICA\\
{\large\bf Departamento de Matemática}\\[1mm] \textit{Prof. Lindeval: lindeval.ufrr@gmail.com}\\\end{center}

\noindent Nome: \rule{9cm}{0.3mm}\ \ Matrícula: \rule{3.5cm}{0.3mm}\\[1cm]
\noindent
1) Determine a equação do plano que passa pelo ponto $P(2,-1,3)$ e é perpendicular ao vetor $\vec{n} = (1,2,-1)$.\\
\noindent
2) Seja a reta $r: x = -2 + 3t$, $y = 1 - t$, $z = 4 + 2t$, e o plano $x - 2y + z = 5$. Determine o ponto de interseção.\\
\noindent
3) Escreva a equação geral de um plano.\\
\noindent
4) Determine o ângulo entre a reta $r: x = 1 + t$, $y = -2 - t$, $z = 3t$ e o plano $2x - y + z = 1$.\\
\noindent
5) Dada a reta $r: x = 2t - 1$, $y = 3t + 2$, $z = -t$, e o plano $2x - y + 3z = 4$, determine o ponto de interseção.\\
\vfill\hfill\bf{\textit{Boa Sorte!}}
\end{document}